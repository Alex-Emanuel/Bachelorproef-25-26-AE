%==============================================================================
% Sjabloon onderzoeksvoorstel bachproef
%==============================================================================
% Gebaseerd op document class `hogent-article'
% zie <https://github.com/HoGentTIN/latex-hogent-article>

% Voor een voorstel in het Engels: voeg de documentclass-optie [english] toe.
% Let op: kan enkel na toestemming van de bachelorproefcoördinator!
\documentclass{hogent-article}

% Invoegen bibliografiebestand
\addbibresource{voorstel.bib}

% Informatie over de opleiding, het vak en soort opdracht
\studyprogramme{Professionele bachelor toegepaste informatica}
\course{Bachelorproef}
\assignmenttype{Onderzoeksvoorstel}

\academicyear{2025-2026}

\title{Bewuste keuzes bij streaming: ontwikkeling van een Proof of Concept die jongvolwassenen waarschuwt voor misleidende technieken op streamingdiensten}

\projectrepo{https://github.com/Alex-Emanuel/Bachelorproef-25-26-AE}

\author{Alex Emanuel}
\email{alex.emanuel@student.hogent.be}

% TODO: Geef de co-promotor op
\supervisor[Co-promotor]{G. De Bruyn (KBC Bank & Verzekeringen, \href{mailto:jane.doe@bedrijf.be}{jane.doe@bedrijf.be})}
\supervisor[Co-promotor]{- (\emph{zoektocht lopende})}

\specialisation{Mobile \& Enterprise development}
\keywords{Dark patterns, Deceptive patterns, Manipulative patterns, Streamingdiensten, Abonnementsservices, Bewustwording, UI/UX design}

\begin{document}

\begin{abstract}
  Nu streamingdiensten niet meer weg te denken zijn uit het leven van Vlaamse jongvolwassenen, worden 18- tot 34-jarigen meer dan ooit blootgesteld aan zogenaamde dark patterns. Deze misleidende technieken beïnvloeden hun keuzes zonder dat zij zich hiervan bewustzijn. Deze bachelorproef onderzoekt hoe zo’n patronen het gedrag van Vlaamse jongvolwassenen beïnvloeden en op welke manier een tool kan helpen om deze technieken sneller te herkennen en hun bewustzijn te verhogen. Er wordt specifiek vertrokken vanuit de centrale onderzoeksvraag ``Op welke manier kan het bewustzijn van op de achtergrond beïnvloedende dark patterns op streamingdiensten verhoogd worden bij jongvolwassen Vlamingen tussen 18 en 34 jaar?’’. Dit onderzoek heeft als doel te bepalen welke tool het meest geschikt is en deze uit te werken in de vorm van een Proof of Concept. Dit wordt onderzocht door het doorlopen van zes fasen: uitvoeren van een literatuurstudie, verzamelen van requirements, uitwerken van een longlist en shortlist van mogelijke oplossingen, ontwikkelen en testen van een Proof of Concept, en afsluiten met een conclusie. Er wordt verwacht dat een doordacht uitgewerkte digitale tool een meerwaarde kan bieden bij het sneller herkennen van dark patterns. Deze meerwaarde wordt gemeten aan de hand van herkenningstesten waarbij een beperkte groep gebruikers dark patterns aanduiden, zowel zonder als met gebruik van de tool. Hierbij wordt gekeken naar het aantal correct opgemerkte dark patterns en de herkenningstijd die hiervoor nodig is. De resultaten bieden een eerste indicatie van effectiviteit, maar zijn niet representatief voor de gehele doelgroep.
\end{abstract}

\tableofcontents

% De hoofdtekst van het voorstel zit in een apart bestand, zodat het makkelijk
% kan opgenomen worden in de bijlagen van de bachelorproef zelf.
%---------- Inleiding ---------------------------------------------------------

\section{Inleiding}%
\label{sec:inleiding}

De voorbije jaren wordt er steeds meer gestreamd in België. Populaire streamingdiensten zoals \emph{Netflix}, \emph{Disney+} en \emph{Spotify} zijn haast niet meer weg te denken. Daarnaast springt ook de Vlaamse entertainmentindustrie volop op deze trend. Enkele populaire Vlaamse zenders en productiehuizen kwamen de voorbije jaren met hun eigen platformen. Denk maar aan \emph{Streamz}, \emph{VRT MAX} en \emph{VTM GO}. Door het stijgende aanbod aan streamingdiensten en doordat live televisie in populariteit daalt, sluiten steeds meer Belgen, waaronder Vlamingen, streamingabonnementen af.

\vspace{1em}

Echter wat een groot aantal Belgen niet weet, is dat op veel van deze platformen gebruik wordt gemaakt van \emph{dark patterns}. Het gaat hierbij om misleidende technieken die online bezoekers subtiel beïnvloeden om bepaalde keuzes te maken die ze anders niet zouden maken \autocite{FODEconomie2024}. Voorbeelden hiervan zijn moeilijk vindbare annulatieknoppen, vooraf aangevinkte selectievakjes of het toevoegen van verborgen, ongewenste kosten. Het gebruik van deze patronen is niet alleen onethisch, het maakt online gebruikers ook enorm kwetsbaar doordat ze deze patronen niet opmerken. Daarnaast beseffen ze vaak niet hoe ernstig de gevolgen kunnen zijn, zoals het onbewust afsluiten van abonnementen of het betalen van verborgen extra kosten.

\vspace{1em}

Hoewel veel onderzoek zich richt op bedrijven die \emph{dark patterns} toepassen, zijn er weinig studies die de effecten van deze patronen op gebruikers onder de loep nemen. Daarbovenop ontbreekt onderzoek naar hoe de gebruikers hier beter bewust van kunnen worden. Hieruit volgt onderstaande onderzoeksvraag, waarbij wordt onderzocht hoe een digitale tool ontwikkeld kan worden om gebruikers hierbij te ondersteunen:

\begin{quote}
    \textit{``Welke digitale tool kan ontwikkeld worden om jongvolwassen Vlamingen tussen 21 en 25 jaar te helpen dark patterns op streamingdiensten sneller te herkennen, zodat onbewuste beïnvloeding voorkomen wordt?''}
\end{quote}

Om de doelgroep van dit onderzoek te bepalen, is er allereerst gekeken naar de leeftijdsgroep die het meest actief gebruikmaakt van streamingdiensten in België. Uit de statistieken van \textcite{StatistiekVlaanderen2025} blijkt dat dit vooral jongvolwassenen zijn tussen 18 en 34 jaar. Daarnaast wordt, om het onderzoek haalbaar te maken, de focus gelegd op jongvolwassen, Vlaamse gebruikers tussen 21 en 25 jaar die actief gebruikmaken van streamingdiensten en interesse hebben in het afsluiten van een abonnement.

\vspace{1em}

Op basis van de doelgroep en centrale onderzoeksvraag zijn enkele deelvragen geformuleerd die helpen om het probleem beter te begrijpen en om tot een mogelijke oplossing te komen.

\vspace{1em}

\noindent
\textbf{Deelvragen probleemdomein:}
\begin{enumerate}
    \item Welke specifieke dark patterns komen het meest voor op websites en apps van streamingdiensten die door jongvolwassenen tussen 21 en 25 gebruikt worden?
    \item Hoe beïnvloeden deze dark patterns het beslissingsproces en de gebruikerservaring (vertrouwen, comfort, UX) van jongvolwassenen bij het afsluiten en opzeggen van streaming-abonnementen?
    \item In welke mate zijn jongvolwassenen zich bewust van de aanwezigheid van dark patterns bij het gebruik van streamingdiensten?
    \item Welke verbanden bestaan er tussen de aanwezigheid van specifieke dark patterns en de businessdoelstellingen van streaming-\newline diensten?
    \item Op welk platform (website of mobiele app) komen dark patterns het vaakst voor en waar kan een digitale tool de grootste impact hebben op gebruikersbewustzijn?
\end{enumerate}

\vspace{1em}

\noindent
\textbf{Deelvragen oplossingsdomein:}
\begin{enumerate}
    \item Welke wet- en regelgeving, richtlijnen of aanbevelingen bestaan er rond het gebruik van dark patterns in België en de Europese Unie?
    \item Welke bestaande tools die gebruikers waarschuwen voor dark patterns bestaan er al, en hoe effectief zijn deze in het verhogen van bewustzijn?
    \item Op welke manier kunnen jongvolwassenen het best ondersteund worden bij het herkennen van dark patterns zonder dat hun gebruikerservaring negatief beïnvloed wordt?
    \item Welke digitale of interactieve oplossing kan het meest bijdragen aan het verhogen van bewustzijn en het verbeteren van de gebruikerservaring, en welke functionele en niet-functionele vereisten zijn hiervoor essentieel?
    \item In welke mate verhoogt het gebruik van de ontwikkelde digitale tool het bewustzijn van gebruikers over dark patterns, en welke verbeterpunten kunnen worden geïdentificeerd op basis van gebruikersfeedback?
\end{enumerate}

\vspace{1em}

Hieruit voortvloeiend is het uiteindelijke doel van dit onderzoek te onderzoeken welke digitale tool het meest geschikt is om gebruikers attent te maken op dark patterns, zodat hun bewustzijn hierover wordt vergroot. Als concrete eindresultaat van het onderzoek wordt, naast de scriptie, een Proof-Of-Concept (POC) uitgewerkt in de vorm van een tool die jongvolwassen Vlamingen tussen 21 en 25 jaar tijdens het gebruik van een streamingplatform bewust maakt van en waarschuwt voor de aanwezigheid van dark patterns. Daarnaast moet de tool hen helpen bij het maken van bewustere keuzes, bijvoorbeeld bij het afsluiten en opzeggen van abonnementen.

%Waarover zal je bachelorproef gaan? Introduceer het thema en zorg dat volgende zaken zeker duidelijk aanwezig zijn:
%
%\begin{itemize}
%  \item kaderen thema
%  \item de doelgroep
%  \item de probleemstelling en (centrale) onderzoeksvraag
%  \item de onderzoeksdoelstelling
%\end{itemize}
%
%Denk er aan: een typische bachelorproef is \textit{toegepast onderzoek}, wat betekent dat je start vanuit een concrete probleemsituatie in bedrijfscontext, een \textbf{casus}. Het is belangrijk om je onderwerp goed af te bakenen: je gaat voor die \textit{ene specifieke probleemsituatie} op zoek naar een goede oplossing, op basis van de huidige kennis in het vakgebied.
%
%De doelgroep moet ook concreet en duidelijk zijn, dus geen algemene of vaag gedefinieerde groepen zoals \emph{bedrijven}, \emph{developers}, \emph{Vlamingen}, enz. Je richt je in elk geval op it-professionals, een bachelorproef is geen populariserende tekst. Eén specifiek bedrijf (die te maken hebben met een concrete probleemsituatie) is dus beter dan \emph{bedrijven} in het algemeen.
%
%Formuleer duidelijk de onderzoeksvraag! De begeleiders lezen nog steeds te veel voorstellen waarin we geen onderzoeksvraag terugvinden.
%
%Schrijf ook iets over de doelstelling. Wat zie je als het concrete eindresultaat van je onderzoek, naast de uitgeschreven scriptie? Is het een proof-of-concept, een rapport met aanbevelingen, \ldots Met welk eindresultaat kan je je bachelorproef als een succes beschouwen?

%---------- Stand van zaken ---------------------------------------------------

\section{Literatuurstudie}%
\label{sec:literatuurstudie}

Hier beschrijf je de \emph{state-of-the-art} rondom je gekozen onderzoeksdomein, d.w.z.\ een inleidende, doorlopende tekst over het onderzoeksdomein van je bachelorproef. Je steunt daarbij heel sterk op de professionele \emph{vakliteratuur}, en niet zozeer op populariserende teksten voor een breed publiek. Wat is de huidige stand van zaken in dit domein, en wat zijn nog eventuele open vragen (die misschien de aanleiding waren tot je onderzoeksvraag!)?

Je mag de titel van deze sectie ook aanpassen (literatuurstudie, stand van zaken, enz.). Zijn er al gelijkaardige onderzoeken gevoerd? Wat concluderen ze? Wat is het verschil met jouw onderzoek?

Verwijs bij elke introductie van een term of bewering over het domein naar de vakliteratuur, bijvoorbeeld~\autocite{Hykes2013}! Denk zeker goed na welke werken je refereert en waarom.

Draag zorg voor correcte literatuurverwijzingen! Een bronvermelding hoort thuis \emph{binnen} de zin waar je je op die bron baseert, dus niet er buiten! Maak meteen een verwijzing als je gebruik maakt van een bron. Doe dit dus \emph{niet} aan het einde van een lange paragraaf. Baseer nooit teveel aansluitende tekst op eenzelfde bron.

Als je informatie over bronnen verzamelt in JabRef, zorg er dan voor dat alle nodige info aanwezig is om de bron terug te vinden (zoals uitvoerig besproken in de lessen Research Methods).

%Dit is een testalinea met mijn autocite poging :)~\autocite{BongardBlanchy2022}
%
%Dit is een testalinea met mijn autocite poging :) \autocite{BongardBlanchy2022}
%
%Dit is voor als ik het wil gebruiken based op een persoon \textcite{BongardBlanchy2022}

% Voor literatuurverwijzingen zijn er twee belangrijke commando's:
% \autocite{KEY} => (Auteur, jaartal) Gebruik dit als de naam van de auteur
%   geen onderdeel is van de zin.
% \textcite{KEY} => Auteur (jaartal)  Gebruik dit als de auteursnaam wel een
%   functie heeft in de zin (bv. ``Uit onderzoek door Doll & Hill (1954) bleek
%   ...'')

Je mag deze sectie nog verder onderverdelen in subsecties als dit de structuur van de tekst kan verduidelijken.

%---------- Methodologie ------------------------------------------------------
\section{Methodologie}%
\label{sec:methodologie}

Hier beschrijf je hoe je van plan bent het onderzoek te voeren. Welke onderzoekstechniek ga je toepassen om elk van je onderzoeksvragen te beantwoorden? Gebruik je hiervoor literatuurstudie, interviews met belanghebbenden (bv.~voor requirements-analyse), experimenten, simulaties, vergelijkende studie, risico-analyse, PoC, \ldots?

Valt je onderwerp onder één van de typische soorten bachelorproeven die besproken zijn in de lessen Research Methods (bv.\ vergelijkende studie of risico-analyse)? Zorg er dan ook voor dat we duidelijk de verschillende stappen terug vinden die we verwachten in dit soort onderzoek!

Vermijd onderzoekstechnieken die geen objectieve, meetbare resultaten kunnen opleveren. Enquêtes, bijvoorbeeld, zijn voor een bachelorproef informatica meestal \textbf{niet geschikt}. De antwoorden zijn eerder meningen dan feiten en in de praktijk blijkt het ook bijzonder moeilijk om voldoende respondenten te vinden. Studenten die een enquête willen voeren, hebben meestal ook geen goede definitie van de populatie, waardoor ook niet kan aangetoond worden dat eventuele resultaten representatief zijn.

Uit dit onderdeel moet duidelijk naar voor komen dat je bachelorproef ook technisch voldoen\-de diepgang zal bevatten. Het zou niet kloppen als een bachelorproef informatica ook door bv.\ een student marketing zou kunnen uitgevoerd worden.

Je beschrijft ook al welke tools (hardware, software, diensten, \ldots) je denkt hiervoor te gebruiken of te ontwikkelen.

Probeer ook een tijdschatting te maken. Hoe lang zal je met elke fase van je onderzoek bezig zijn en wat zijn de concrete \emph{deliverables} in elke fase?

%---------- Verwachte resultaten ----------------------------------------------
\section{Verwacht resultaat, conclusie}%
\label{sec:verwachte_resultaten}

Hier beschrijf je welke resultaten je verwacht. Als je metingen en simulaties uitvoert, kan je hier al mock-ups maken van de grafieken samen met de verwachte conclusies. Benoem zeker al je assen en de onderdelen van de grafiek die je gaat gebruiken. Dit zorgt ervoor dat je concreet weet welk soort data je moet verzamelen en hoe je die moet meten.

Wat heeft de doelgroep van je onderzoek aan het resultaat? Op welke manier zorgt jouw bachelorproef voor een meerwaarde?

Hier beschrijf je wat je verwacht uit je onderzoek, met de motivatie waarom. Het is \textbf{niet} erg indien uit je onderzoek andere resultaten en conclusies vloeien dan dat je hier beschrijft: het is dan juist interessant om te onderzoeken waarom jouw hypothesen niet overeenkomen met de resultaten.



\printbibliography[heading=bibintoc]

\end{document}