%---------- Inleiding ---------------------------------------------------------

\section{Inleiding}%
\label{sec:inleiding}

Er wordt steeds meer gestreamd in België \autocite{BNA2025}. Populaire streamingdiensten zoals \emph{Netflix}, \emph{Disney+} en \emph{Spotify} zijn haast niet meer weg te denken. Daarnaast springt ook de Vlaamse entertainmentindustrie volop op deze trend. Enkele populaire Vlaamse zenders en productiehuizen kwamen de voorbije jaren met hun eigen platformen. Denk maar aan \emph{Streamz}, \emph{VRT MAX} en \emph{VTM GO}. Door de groei van streamingdiensten en doordat live televisie in populariteit daalt \autocite{Digimeter2024}, maken steeds meer Belgen, waaronder Vlamingen, gebruik van al dan niet betalende streamingdiensten \autocite{BNA2025}.

\vspace{1em}

Echter wat een groot aantal Belgen niet weet, is dat veel van deze platformen gebruikmaken van dark patterns. Het gaat hierbij om misleidende technieken die online gebruikers subtiel beïnvloeden om bepaalde keuzes te maken die ze anders niet zouden maken. Voorbeelden hiervan zijn onder andere moeilijk vindbare annulatieknoppen of vooraf aangevinkte selectievakjes \autocite{FODEconomie2024}. 

Niet enkel is het gebruik van deze patronen ethisch onverantwoord, het maakt gebruikers ook enorm kwetsbaar doordat ze deze patronen onvoldoende of niet opmerken. Daarnaast wordt de ernst van de gevolgen vaak onderschat, zoals het onbewust afsluiten van een abonnement of het betalen van een verborgen extra kost.

\vspace{1em}

Ondanks dat er veel onderzoek werd gedaan naar dark patterns vanuit het standpunt van zij die er gebruik van maken, zijn er weinig studies die de effecten van deze patronen op de gebruikers (lees \emph{slachtoffers}) onder de loep nemen. Daarbovenop is er nood aan onderzoek naar hoe de gebruikers hier meer bewust van kunnen worden. Uit deze bevindingen werd onderstaande onderzoeksvraag opgesteld:

\vspace{1em}

\begin{quote}    
    \textit{``Op welke manier kan het bewustzijn van op de achtergrond beïnvloedende dark patterns op streamingdiensten verhoogd worden bij jongvolwassen Vlamingen tussen 18 en 34 jaar?''}
\end{quote}

\vspace{1em}

Om de doelgroep van dit onderzoek te bepalen is er gekeken naar de leeftijdsgroep die het meest actief gebruikmaakt van streamingdiensten in België. Uit de statistieken van \textcite{StatistiekVlaanderen2025} blijkt dat dit vooral jongvolwassenen zijn tussen 18 en 34 jaar. Hierdoor ligt de focus voor dit onderzoek op Vlaamse gebruikers tussen 18 en 34 jaar die actief gebruikmaken van streamingdiensten en interesse tonen in het afsluiten van een abonnement.

\vspace{1em}

Aanvullend werden enkele deelvragen geformuleerd die ondersteuning bieden bij het beter begrijpen van het probleem en het zoeken naar een mogelijke oplossing.

\vspace{2em}

\noindent
\textbf{Deelvragen probleemdomein:}
\begin{enumerate}
    \item Welke dark patterns komen het meest voor op websites en apps van streamingdiensten die door jongvolwassenen tussen 18 en 34 gebruikt worden?
    \item In welke mate zijn jongvolwassenen zich bewust van de aanwezigheid van dark patterns bij het gebruik van streamingdiensten?
    \item Welke verbanden bestaan er tussen de aanwezigheid van specifieke dark patterns en de businessdoelstellingen van streaming-\newline diensten?
    \item Op welk platform (website of mobiele app) komen dark patterns het vaakst voor?
    \item Kunnen technologische tools helpen bij het verhogen van bewustzijn rond bepaalde onderwerpen?
\end{enumerate}

\noindent
\textbf{Deelvragen oplossingsdomein:}
\begin{enumerate}
    \item Welke wet- en regelgeving, richtlijnen of aanbevelingen bestaan er rond het gebruik van dark patterns in België en de Europese Unie?
    \item Welke bestaande tools die gebruikers waarschuwen voor dark patterns bestaan er al, en hoe effectief zijn deze in het verhogen van bewustzijn?
    \item Op welke manier kunnen jongvolwassenen het best ondersteund worden bij het herkennen van dark patterns zonder dat hun gebruikerservaring negatief beïnvloed wordt?
    \item Welke digitale of interactieve oplossing kan het meest bijdragen aan het verhogen van bewustzijn, en welke (niet-)functionele vereisten zijn hiervoor essentieel?
    \item In welke mate voldoet de uitgewerkte digitale oplossing aan de opgestelde requirements en welke verbeterpunten kunnen worden geïdentificeerd?
\end{enumerate}

\vspace{1em}

Als uiteindelijk doel van dit onderzoek wordt er achterhaald welke digitale tool het meest geschikt is om gebruikers tijdens het gebruik van streamingplatformen meer attent te maken op de aanwezigheid van dark patterns. Naast de scriptie wordt als concreet eindresultaat eveneens een Proof of Concept (PoC) uitgewerkt die getest wordt bij een beperkte groep jongvolwassenen (tussen 18 en 34 jaar) om inzicht te krijgen in het effect van de tool op het bewustzijn van en het sneller herkennen van dark patterns. De resultaten van deze testen geven een eerste indicatie over de effectiviteit van de tool, maar zijn nog niet representatief voor de gehele doelgroep.

\vspace{1em}

%---------- Stand van zaken ---------------------------------------------------

\section{Literatuurstudie}%
\label{sec:literatuurstudie}

\subsection{Definitie van dark patterns}
\label{subsec:definitie}

\vspace{1em}

Voor online verkoop een prominente rol speelde op de handelsmarkt, werden volgens \textcite{Kollmer2022} technieken voor manipulatie en misleiding voornamelijk toegepast in fysieke winkels. Echter werden deze methodes door de digitalisering ook overgebracht naar de wereld van e-commerce. Hierdoor ontstonden de zogenaamde dark patterns. Deze term deed dankzij Harry Brignull in 2010 zijn intrede, die deze patronen omschreef als \textbf{trucs} gebruikt op websites en apps \textbf{om gebruikers tegen hun wil dingen te laten doen}, zoals iets aankopen of zich ergens voor inschrijven \autocite{Kollmer2022}. 

\vspace{1em}

Bovendien kunnen deze patronen, zoals beschreven door \textcite{DiGeronimo2020}, verder onderverdeeld worden in vijf categorieën. Ten eerste is er \emph{nagging}, het herhaaldelijk vragen aan de gebruiker om iets anders te doen. Ten tweede \emph{obstruction}, het opzettelijk moeilijker maken van het uitvoeren van acties. Vervolgens bestaat \emph{sneaking}, het verbergen van relevante informatie zoals bijvoorbeeld verborgen kosten. Daarnaast is er sprake van \emph{interface interference}: het verwarrend, misleidend of slecht leesbaar maken van user interface elementen. Tot slot is er \emph{forced action} waarbij de gebruiker gedwongen wordt om ongewenste keuzes te maken.

\vspace{1em}

\subsection{Gebruik van dark patterns op digitale platformen}
\label{subsec:verspreiding}

\vspace{1em}

Uit een rapport van het \textcite{ICPEN2024} is gebleken dat dark patterns voorkomen in een wijde selectie aan sectoren. Denk onder meer aan de entertainment-, streaming-, fitness- en e-commerce industrie. Eveneens bleek uit dit onderzoek dat zo’n 75,7\% van apps en websites die abonnementen verkopen minstens één dark pattern gebruiken. Aanvullend werd er geconstateerd dat bij 66,8\% van deze onderzochte websites en apps twee of meer patronen terug te vinden zijn. Verder concludeerden \textcite{Nygren2024} dat binnen de streamingsector, zowel de gratis als de te betalen platformen, uitgebreid gebruikmaken van deze patronen. Ook werd dankzij de studie van \textcite{Gunawan2021} duidelijk dat zowel de desktop- als mobiele versies dark patterns bevatten, maar dat de versies van éénzelfde dienst daarvoor niet meteen dezelfde patronen inzetten.


\vspace{1em}

\subsection{Bewustzijn van dark patterns en de effecten op gebruikersgedrag}
\label{subsec:effecten}

\vspace{1em}

Zoals de definitie van een dark pattern aanhaalt is zo’n patroon bedoelt om de gebruiker te manipuleren \autocite{Kollmer2022}. Maar wat is de invloed hiervan op het bewustzijn en gebruikersgedrag? \textcite{OECD2022} onderzocht dit en constateerde dat de meeste gebruikers moeite hebben bij het herkennen van deze technieken, terwijl sommige gebruikers ze wel opmerken maar er toch slachtoffer van worden. Volgens \textcite{BongardBlanchy2021} is een aanzienlijk deel van de gebruikers op de hoogte van het gebruik van deze patronen en verwachten ze het bij bekende diensten zoals bijvoorbeeld Amazon en Netflix. Echter beseffen ze niet de mate waarin deze patronen schade kunnen aanrichten. Ook zij halen aan dat het herkennen lastig blijft, waardoor gebruikers hun nog altijd laten beïnvloeden, zelfs als ze weten dat deze praktijken worden toegepast. Hieruit konden ze concluderen dat gebruikers zich niet bewust zijn van hoe deze ontwerpkeuzes hun concreet kunnen schaden \autocite{BongardBlanchy2021}. 

\vspace{1em}

Aanvullend werd door \textcite{OECD2022} onderzocht welke patronen het gedrag het meest beïnvloeden. Hieruit konden ze 2 types onderscheiden: de “\emph{mild}” en “\emph{aggressive}” dark patterns. Volgens hen zijn de “\emph{mild}” dark patterns (de minder opvallende patronen) vaak effectiever doordat ze zo moeilijk op te merken zijn. Verder werd door \textcite{OECD2022} onderzocht wat de effecten zijn op de gebruiker. Zo schaadt het de autonomie, kan het leiden tot financieel verlies, als ook zorgen de technieken voor privacy schending en kunnen gebruikers emotionele stress en tijdverlies ervaren. 


\vspace{1em}

\subsection{De ethiek achter het gebruik van dark patterns}
\label{subsec:ethisch-ontwerp}

\vspace{1em}

Hoewel niet alle dark patterns manipulatief zijn \autocite{Kitkowska2023}, moet er  nagedacht worden over de ethiek die hierachter schuilgaat. En hoe ethisch is ontwerp in het algemeen? \textcite{DiGeronimo2020} stelt dat een bruikbare applicatie niet per se duidt op een ethische applicatie. Volgens hen hebben ontwerpers de rol om met hun ontwerp de gebruiker te overtuigen, echter wordt die macht vaak misbruikt. Dit uit zich onder andere in de vorm van dark patterns. Tot dusver is er volgens deze studie nog geen éénduidige definitie voor wat een ethische user interface juist inhoudt, echter hebben de meeste experts een gelijkaardige mening: een user interface wordt als ethisch beschouwd wanneer het ontwerp meer bijdraagt tot het leven van de gebruiker dan dat het de gebruiker schaadt. Dit brengt echter volgende vraag met zich mee: \emph{``Op welk punt gaat een ontwerp van het ondersteunen over naar het misleiden en/of misbruiken van de gebruiker?''}. Tot slot gaven \textcite{Vandenberghe2016} nog een belangrijk inzicht mee: stel niet meteen user-centered design gelijk aan ethisch ontwerp. Bepaalde methoden kunnen namelijk nog steeds schadelijke effecten hebben. Volgens hen bestaat er een \emph{Ethical Design Manifesto} dat benadrukt dat een ontwerp niet enkel moet verleiden of esthetisch zijn, maar het gebruikers ook kritisch moet laten nadenken over de impact van technologie op hun leven.

\vspace{1em}

\subsection{Onderzoek en tools met betrekking tot dark patterns}
\label{subsec:huidig-onderzoek}

\vspace{1em}

In het algemeen werd er in het kader van dark patterns voornamelijk onderzocht welke verschillende soorten dark patterns bestaan en waar deze het frequentst voorkomen \autocite{Luguri2021}. Naast deze studies is er volgens \textcite{Luguri2021} nog te weinig onderzoek gedaan op andere gebieden binnen dit domein. Ook de studie van \textcite{DiGeronimo2020} benadrukt de nood naar verder onderzoek. Volgens hen is het belangrijk dat de focus wordt gelegd op het vinden van manieren om bewustwording bij gebruikers te stimuleren, onder meer via de ontwikkeling van geautomatiseerde en educatieve tools. Hoewel er volgens \textcite{Nie2024} al redelijk wat detectietools bestaan die door middel van AI dark patterns opsporen, bieden deze een te lage dekking.  \textcite{Nie2024} identificeerden maar liefst 64 types dark patterns, waarvan slechts de helft werd gedetecteerd door de acht onderzochte geautomatiseerde tools. 

\vspace{1em}

%---------- Methodologie ------------------------------------------------------

\begin{figure*}
    \centering
    \includegraphics[width=\textwidth]{../graphics/gantt}
    \caption{\label{fig:gantt}Gantt diagram met de verschillende fasen van het onderzoek}
\end{figure*}

\section{Methodologie}%
\label{sec:methodologie}

Zoals in het begin van dit voorstel werd aangehaald, richt dit onderzoek zich op het zoeken naar een geschikte tool die jongvolwassen gebruikers in staat stelt om dark patterns op streamingdiensten te herkennen, en zo bewustwording rond dit onderwerp te creëren. Om de onderzoeksvraag te beantwoorden, wordt een weloverwogen plan gevolgd dat bestaat uit \textbf{vijf verschillende fasen}. In eerste instantie wordt de huidige stand van zaken onderzocht aan de hand van een literatuurstudie en zal een longlist gemaakt worden van alle mogelijke oplossingen. Nadien worden uit het literatuuronderzoek de requirements waaraan de tool moet voldoen opgesteld. Aan de hand van deze vereisten zal de longlist uit fase 1 verder vernauwd worden tot een shortlist die de meest geschikte tools bevat voor deze casus. Hieruit zal de meest beloftevolle aanpak gekozen en uitgewerkt worden als een Proof of Concept door een bestaande open‑source tool verder uit te werken en te verbeteren voor deze specifieke use case. Tot slot zal uit dit onderzoek een conclusie geformuleerd worden. Naast deze gedefinieerde fasen wordt er tijdens het volledige verloop van het onderzoek continu documentatie bijgehouden. Iedere fase is zorgvuldig gepland om een optimaal verloop van de bachelorproef te garanderen. Een overzicht van het verloop van het onderzoek wordt weergegeven in Figuur~\ref{fig:gantt}. De onderzoeksperiode spreidt zich over \textbf{11 weken}, gevolgd door 3 weken waarin feedback kan worden verwerkt en de eindpresentatie kan worden voorbereid. Binnen de onderzoeksperiode zal er \textbf{wekelijks één tot twee dagen} gewerkt worden aan de bachelorproef: op de bachelorproefdag en, indien mogelijk, één dag in het weekend.

\vspace{1em}

\subsection{Fase 1: Stand van zaken \& Longlist}
\label{subsec:stand-van-zaken}

\begin{tabular}{ll}
    \hspace{0.3em} \textbf{Periode:} & 16/02 – 08/03 (Week 1 - 3)
\end{tabular}

\vspace{1em}

Tijdens de eerste twee weken van het onderzoek wordt het onderzoeksdomein uitvoerig verkend om op die manier een goed theoretisch kader op te bouwen. Via literatuuronderzoek wordt onderzocht wat dark patterns zijn, welke er bestaan, welke negatieve gevolgen deze kunnen hebben en hoe bewust mensen zijn van de patronen en hun gevolgen. Verder wordt nagegaan waar deze patronen het vaakst voorkomen (websites en apps) en of er verschillen zijn per platform. Daarnaast wordt ook de huidige wet- en regelgeving en de mogelijk link tussen de businessdoelstellingen van streamingdiensten en het gebruik van dark patterns onder de loep genomen. Aanvullend aan dit literatuuronderzoek, wordt in kaart gebracht welke streamingdiensten het populairst zijn in België en welke dark patterns zij toepassen, onderverdeeld per platform. Tot slot worden bestaande tools geanalyseerd om hun sterktes en beperkingen te identificeren. Het literatuuronderzoek zal gedaan worden gebruikmakend van \emph{Google Scholar} en wetenschappelijke databanken zoals \emph{ACM digital library}, \emph{Springer Online Journals} en \emph{JSTOR}.

\vspace{1em}

Op basis van dit literatuuronderzoek wordt een \textbf{longlist} opgesteld van mogelijke digitale oplossingen die gebruikers kunnen helpen om dark patterns op streamingdiensten te herkennen. Ieder alternatief wordt gedocumenteerd met zijn voor- en nadelen. Het doel is om een zo compleet mogelijk overzicht te creëren van mogelijke oplossingen, dat later gebruikt zal worden voor het opstellen van de shortlist.

\vspace{1em}

Het doel van deze fase is het verder verwerven van kennis omtrent het probleemdomein, zodat in de volgende fase de requirements op een onderbouwde manier kunnen worden opgesteld. Dit resulteert zich in het uitwerken van een overzicht van gebruikersproblemen rond dark patterns op streamingdiensten, een lijst van populaire streamingdiensten en de patronen die ze bevatten, en een analyse van bestaande tools met hun sterke en zwakke punten. Tot slot moet in deze fase een longlist opgesteld zijn van alle mogelijke digitale oplossingen die geschikt zouden kunnen zijn om in een verder stadium uit te werken.

\vspace{1em}

\subsection{Fase 2: Requirements verzamelen}
\label{subsec:requirements-verzamelen}

\begin{tabular}{ll}
    \hspace{0.3em} \textbf{Periode:} & 09/03 – 15/03 (Week 4)
\end{tabular}

\vspace{1em}

In fase~2 ligt de focus op het verzamelen van alle vereisten waaraan de nieuwe tool moet voldoen. Op basis van de resultaten uit het literatuuronderzoek worden de requirements opgesteld. Er wordt onder andere rekening gehouden met welke dark patterns het belangrijkste zijn, waar de tool moet ingezet worden, welke aspecten overgenomen kunnen worden van andere tools en welke toevoegingen nodig zijn voor een betere digitale oplossing. Aanvullend worden de voorlopige requirements afgetoetst en aangevuld via interviews met een UI/UX-designer en een front-end developer. Deze fase zal worden afgerond door het opstellen van een volledige en geprioriteerde lijst van criteria die het \emph{minimum viable product (MVP)} definiëren. Het concrete resultaat bestaat uit een lijst van functionele en niet-functionele requirements, geordend volgens hun belang. Voor de ordening zal de \emph{MoSCoW}-methode gebruikt worden.

\vspace{1em}

\subsection{Fase 3: Shortlist}
\label{subsec:shortlist}

\begin{tabular}{ll}
    \hspace{0.3em} \textbf{Periode:} & 16/03 – 22/03 (Week 5)
\end{tabular}

\vspace{1em}

Vooraleer een Proof of Concept uitgewerkt kan worden, wordt er gekeken welke oplossing uit de longlist het meest beloftevol en waardevol is voor het onderzoek. Daarvoor worden in een vierde fase alle alternatieven systematisch met elkaar vergeleken op basis van de functionele en niet-functionele requirements uit fase~2 door middel van een overzichtelijke vergelijkingstabel. Hieruit wordt de beste optie gekozen. Indien één of meerdere alternatieven dicht bij elkaar scoren, worden bijkomende criteria zoals voor- en nadelen en de verwachte impact op de gebruiker ook in rekening gebracht. Op basis van deze analyse zal een shortlist worden opgesteld, waaruit uiteindelijk de meest beloftevolle oplossing geselecteerd zal worden. Tot slot wordt de keuze voor deze tool grondig onderbouwd door middel van een duidelijke motivatie. De tool die uit deze analyse als meest geschikt wordt bevonden, zal in de voorlaatste fase verder uitgewerkt worden.

\vspace{1em}

\subsection{Fase 4: Proof of Concept}
\label{subsec:proof-of-concept}

\begin{tabular}{ll}
    \hspace{0.3em} \textbf{Periode:} & 23/03 – 26/04 (Week 6 - 10)
\end{tabular}

\vspace{1em}

Fase~4, de fase die het meeste tijd in beslag zal nemen, omvat de uitwerking van een Proof of Concept (PoC). De geselecteerde oplossing uit fase~4 wordt in deze fase uitgewerkt tot een eerste Minimum Viable Product (MVP). Aangezien er reeds bestaande open-source tools beschikbaar zijn, zal één van deze tools verder aangepast en uitgebreid worden voor de specifieke use case van dit onderzoek. 

\vspace{1em}

In eerste instantie zullen er wireframes en mockups uitgewerkt worden in \emph{AdobeXD} om de algemene flow van de tool te bepalen. Nadien zullen in overleg met de co-promotoren de meest geschikte technologieën (zoals \emph{React}, \emph{Node.js}, \emph{Python}, \ldots) gekozen worden. Daarna start de implementatie van de tool. Tijdens de ontwikkeling wordt er iteratief en agile gewerkt, waarbij het resultaat van elke iteratie zal worden afgetoetst bij de co-promotoren. Daarnaast zullen er testen geschreven worden die de goede werking van de tool waarborgen.

\vspace{1em}

Op het einde van de fase is er een werkend prototype die de belangrijkste functionaliteiten van de tool bevat, aangevuld met een korte documentatie over de werking en functionaliteiten. Eveneens zal gevalideerd worden of de uitgewerkte tool al dan niet een oplossing biedt voor het gestelde probleem.
Dit zal gedaan worden aan de hand van een digitale test bij enkele testgebruikers die tot de doelgroep behoren. Zij zullen een aantal taken uitvoeren op een streamingplatform dat gebruikmaakt van dark patterns, zowel zonder als met de tool. Uit deze test wordt een eerste indicatie afgeleid van in welke mate de proefpersonen de patronen herkennen en hoe snel zij deze waarnemen. De resultaten hiervan geven een eerste indicatie in de mogelijke meerwaarde van de PoC, zonder ervan uit te gaan dat dit geldt voor de hele doelgroep.

\vspace{1em}

\subsection{Fase 5: Conclusie}
\label{subsec:conclusie}

\begin{tabular}{ll}
    \hspace{0.3em} \textbf{Periode:} & 20/04 – 03/05 (Week 10 - 11)
\end{tabular}

\vspace{1em}

De laatste fase omvat het trekken van een conclusie over de werking van de uitgewerkte tool en het identificeren van verbeterpunten en overblijvende hiaten. Dit gebeurt door de gerealiseerde functionaliteiten te vergelijken met de vereisten die in fase~2 werden opgesteld. Daarnaast worden observaties uit de gebruikerstesten meegenomen om hiaten te identificeren en mogelijke verbeterpunten te formuleren.

\vspace{1em}


%---------- Verwachte resultaten ----------------------------------------------
\section{Verwacht resultaat}%
\label{sec:verwachte_resultaten}

Het verwachte resultaat van dit onderzoek is een digitale tool die Vlamingen tussen 18 en 34 jaar helpt bij het opmerken van dark patterns op streamingdiensten zodat ze deze sneller herkennen. De tool zal hen ondersteunen bij het opmerken van misleidende technieken en zal de gebruikers informeren over de intentie en gevaren ervan. Deze verwachting baseert zich op bevindingen uit de literatuurstudie, namelijk dat gebruikers regelmatig moeite hebben bij het herkennen van dark patterns \autocite{OECD2022} en dat ze niet bewust zijn van de gevaren ervan \autocite{BongardBlanchy2021}. Daarnaast wordt verwacht dat de tool geen AI zal gebruiken aangezien uit de literatuurstudie gebleken is dat bestaande detectietools gebruikmakend van AI een beperkte dekking hebben \autocite{Nie2024} en dus onmogelijk alle patronen kunnen herkennen en ervoor waarschuwen. Eveneens wordt, naast de uitgewerkte tool, aangenomen dat het onderzoek inzichten zal geven in op welke manier gebruikers het best geïnformeerd en gewaarschuwd kunnen worden. 

\vspace{1em}

De verwachte meerwaarde van dit onderzoek voor de doelgroep ligt in het verhogen van hun bewustzijn over dark patterns en het vergroten van hun gevoel van controle bij het gebruik van streamingplatformen. De tool ondersteunt gebruikers bij het maken van bewustere keuzes door hen te laten zien hoe interfaces hen kunnen beïnvloeden. Verder biedt dit onderzoek waardevolle inzichten voor de wereld van UI/UX en softwareontwikkeling, doordat het kan dienen als voorbeeld voor het ontwerpen van soortgelijke tools in andere domeinen waar dark patterns voorkomen, zoals webshops of reiswebsites.

\vspace{1em}