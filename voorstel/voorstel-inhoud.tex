%---------- Inleiding ---------------------------------------------------------

\section{Inleiding}%
\label{sec:inleiding}

De voorbije jaren wordt er steeds meer gestreamd in België. Populaire streamingdiensten zoals \emph{Netflix}, \emph{Disney+} en \emph{Spotify} zijn haast niet meer weg te denken. Daarnaast springt ook de Vlaamse entertainmentindustrie volop op deze trend. Enkele populaire Vlaamse zenders en productiehuizen kwamen de voorbije jaren met hun eigen platformen. Denk maar aan \emph{Streamz}, \emph{VRT MAX} en \emph{VTM GO}. Door het stijgende aanbod aan streamingdiensten en doordat live televisie in populariteit daalt, sluiten steeds meer Belgen, waaronder Vlamingen, streamingabonnementen af.

\vspace{1em}

Echter wat een groot aantal Belgen niet weet, is dat veel van deze platformen gebruikmaken van dark patterns. Het gaat hierbij om misleidende technieken die online gebruikers subtiel beïnvloeden om bepaalde keuzes te maken die ze anders niet zouden maken \autocite{FODEconomie2024}. Voorbeelden hiervan kunnen onder andere moeilijk vindbare annulatieknoppen of vooraf aangevinkte selectievakjes zijn. Niet enkel is het gebruik van deze patronen ethisch onverantwoord, het maakt gebruikers ook enorm kwetsbaar doordat ze deze patronen onvoldoende of niet opmerken. Daarnaast wordt de ernst van de gevolgen vaak onderschat, zoals het onbewust afsluiten van een abonnement of het betalen van een verborgen extra kost.

\vspace{1em}

Ondanks dat er veel onderzoek werd gedaan naar dark patterns vanuit het standpunt van zij die er gebruik van maken, zijn er weinig studies die de effecten van deze patronen op de gebruikers (lees \emph{slachtoffers}) onder de loep nemen. Daarbovenop is er nood aan onderzoek naar hoe de gebruikers hier meer bewust van kunnen worden. Uit deze bevindingen werd onderstaande onderzoeksvraag opgesteld:

\vspace{1em}

\begin{quote}
    \textit{``Welke digitale tool kan ontwikkeld worden om jongvolwassen Vlamingen tussen 21 en 25 jaar te helpen dark patterns op streamingdiensten sneller te herkennen, zodat onbewuste beïnvloeding voorkomen wordt?''}
\end{quote}

\vspace{1em}

Om de doelgroep van dit onderzoek te bepalen is er in eerste instantie gekeken naar de leeftijdsgroep die het meest actief gebruikmaakt van streamingdiensten in België. Uit de statistieken van \textcite{StatistiekVlaanderen2025} blijkt dat dit vooral jongvolwassenen zijn tussen 18 en 34 jaar. Daarnaast is de doelgroep verder vernauwd, om het onderzoek haalbaarder te maken. Hierdoor werd de focus gelegd op Vlaamse gebruikers tussen 21 en 25 jaar die actief gebruikmaken van streamingdiensten en interesse tonen in het afsluiten van een abonnement.

\vspace{1em}

Verder werden enkele deelvragen geformuleerd die ondersteuning bieden bij het beter begrijpen van het probleem en het zoeken naar een mogelijke oplossing.

\vspace{1em}

\noindent
\textbf{Deelvragen probleemdomein:}
\begin{enumerate}
    \item Welke specifieke dark patterns komen het meest voor op websites en apps van streamingdiensten die door jongvolwassenen tussen 21 en 25 gebruikt worden?
    \item Hoe beïnvloeden deze dark patterns het beslissingsproces en de gebruikerservaring (vertrouwen, comfort, UX) van jongvolwassenen bij het afsluiten en opzeggen van streaming-abonnementen?
    \item In welke mate zijn jongvolwassenen zich bewust van de aanwezigheid van dark patterns bij het gebruik van streamingdiensten?
    \item Welke verbanden bestaan er tussen de aanwezigheid van specifieke dark patterns en de businessdoelstellingen van streaming-\newline diensten?
    \item Op welk platform (website of mobiele app) komen dark patterns het vaakst voor en waar kan een digitale tool de grootste impact hebben op gebruikersbewustzijn?
\end{enumerate}

\vspace{1em}

\noindent
\textbf{Deelvragen oplossingsdomein:}
\begin{enumerate}
    \item Welke wet- en regelgeving, richtlijnen of aanbevelingen bestaan er rond het gebruik van dark patterns in België en de Europese Unie?
    \item Welke bestaande tools die gebruikers waarschuwen voor dark patterns bestaan er al, en hoe effectief zijn deze in het verhogen van bewustzijn?
    \item Op welke manier kunnen jongvolwassenen het best ondersteund worden bij het herkennen van dark patterns zonder dat hun gebruikerservaring negatief beïnvloed wordt?
    \item Welke digitale of interactieve oplossing kan het meest bijdragen aan het verhogen van bewustzijn en het verbeteren van de gebruikerservaring, en welke functionele en niet-functionele vereisten zijn hiervoor essentieel?
    \item In welke mate verhoogt het gebruik van de ontwikkelde digitale tool het bewustzijn van gebruikers over dark patterns, en welke verbeterpunten kunnen worden geïdentificeerd op basis van gebruikersfeedback?
\end{enumerate}

\vspace{1em}

Als uiteindelijk doel van dit onderzoek wordt er achterhaald welke digitale tool het meest geschikt is om gebruikers meer attent te maken op de aanwezigheid van dark patterns zodat ze hier bewuster van worden. Naast de scriptie wordt als concreet eindresultaat ook een Proof-of-Concept (PoC) uitgewerkt in de vorm van een tool die Vlamingen tussen 21 en 25 jaar tijdens het gebruik van streamingplatformen bewust maakt van en waarschuwt voor de aanwezigheid van dark patterns. Tot slot moet deze tool hen assisteren bij het maken van bewustere, weloverwogen keuzes bij onder meer het afsluiten en opzeggen van abonnementen.

\vspace{1em}

%---------- Stand van zaken ---------------------------------------------------

\section{Literatuurstudie}%
\label{sec:literatuurstudie}

\subsection{Definitie van dark patterns}
\label{subsec:ontstaan-definitie}

\vspace{1em}

Voor online verkoop een prominente rol speelde op de handelsmarkt, werden volgens \textcite{Kollmer2022} technieken voor manipulatie en misleiding voornamelijk toegepast in fysieke winkels. Echter werden deze methodes door de digitalisering ook overgebracht naar de wereld van e-commerce. Hierdoor ontstonden de zogenaamde dark patterns. Deze term deed dankzij Harry Brignull in 2010 zijn intrede, die deze patronen omschreef als \textbf{trucs} gebruikt op websites en apps \textbf{om gebruikers tegen hun wil dingen te laten doen}, zoals iets aankopen of zich ergens voor inschrijven \autocite{Kollmer2022}. 

\vspace{1em}

Bovendien kunnen deze patronen, zoals beschreven door \textcite{DiGeronimo2020}, verder onderverdeeld worden in vijf categorieën. Ten eerste is er \emph{nagging}, het herhaaldelijk vragen aan de gebruiker om iets anders te doen. Ten tweede \emph{obstruction}, het opzettelijk moeilijker maken van het uitvoeren van acties. Vervolgens bestaat \emph{sneaking}, het verbergen van informatie zoals verborgen kosten. Daarnaast is er sprake van \emph{interface interference}: het verwarrend, misleidend of slecht leesbaar maken van user interface elementen. Tot slot is er \emph{forced action} waarbij de gebruiker gedwongen wordt om ongewenste keuzes te maken.

\vspace{1em}

\subsection{Gebruik van dark patterns op \newline digitale platformen}
\label{subsec:verspreiding}

\vspace{1em}

tekst

\vspace{1em}

\subsection{Bewustzijn van dark patterns en de effecten op gebruikersgedrag}
\label{subsec:effecten}

\vspace{1em}

tekst

\vspace{1em}

\subsection{Psychologie en ethiek van dark patterns}
\label{subsec:ethisch-ontwerp}

\vspace{1em}

tekst

\vspace{1em}

\subsection{Bestaand onderzoek naar dark \newline patterns}
\label{subsec:huidig-onderzoek}

\vspace{1em}

tekst

\vspace{1em}

\subsection{Huidige tools rond bewustmaking van dark patterns}
\label{subsec:tools}

\vspace{1em}

tekst

\vspace{1em}

%Hier beschrijf je de \emph{state-of-the-art} rondom je gekozen onderzoeksdomein, d.w.z.\ een inleidende, doorlopende tekst over het onderzoeksdomein van je bachelorproef. Je steunt daarbij heel sterk op de professionele \emph{vakliteratuur}, en niet zozeer op populariserende teksten voor een breed publiek. Wat is de huidige stand van zaken in dit domein, en wat zijn nog eventuele open vragen (die misschien de aanleiding waren tot je onderzoeksvraag!)?
%
%Je mag de titel van deze sectie ook aanpassen (literatuurstudie, stand van zaken, enz.). Zijn er al gelijkaardige onderzoeken gevoerd? Wat concluderen ze? Wat is het verschil met jouw onderzoek?
%
%Verwijs bij elke introductie van een term of bewering over het domein naar de vakliteratuur, bijvoorbeeld~\autocite{Hykes2013}! Denk zeker goed na welke werken je refereert en waarom.
%
%Draag zorg voor correcte literatuurverwijzingen! Een bronvermelding hoort thuis \emph{binnen} de zin waar je je op die bron baseert, dus niet er buiten! Maak meteen een verwijzing als je gebruik maakt van een bron. Doe dit dus \emph{niet} aan het einde van een lange paragraaf. Baseer nooit teveel aansluitende tekst op eenzelfde bron.
%
%Als je informatie over bronnen verzamelt in JabRef, zorg er dan voor dat alle nodige info aanwezig is om de bron terug te vinden (zoals uitvoerig besproken in de lessen Research Methods).

%Dit is een testalinea met mijn autocite poging :)~\autocite{BongardBlanchy2022}
%
%Dit is een testalinea met mijn autocite poging :) \autocite{BongardBlanchy2022}
%
%Dit is voor als ik het wil gebruiken based op een persoon \textcite{BongardBlanchy2022}

% Voor literatuurverwijzingen zijn er twee belangrijke commando's:
% \autocite{KEY} => (Auteur, jaartal) Gebruik dit als de naam van de auteur
%   geen onderdeel is van de zin.
% \textcite{KEY} => Auteur (jaartal)  Gebruik dit als de auteursnaam wel een
%   functie heeft in de zin (bv. ``Uit onderzoek door Doll & Hill (1954) bleek
%   ...'')

%Je mag deze sectie nog verder onderverdelen in subsecties als dit de structuur van de tekst kan verduidelijken.

%---------- Methodologie ------------------------------------------------------
\section{Methodologie}%
\label{sec:methodologie}

Hier beschrijf je hoe je van plan bent het onderzoek te voeren. Welke onderzoekstechniek ga je toepassen om elk van je onderzoeksvragen te beantwoorden? Gebruik je hiervoor literatuurstudie, interviews met belanghebbenden (bv.~voor requirements-analyse), experimenten, simulaties, vergelijkende studie, risico-analyse, PoC, \ldots?

Valt je onderwerp onder één van de typische soorten bachelorproeven die besproken zijn in de lessen Research Methods (bv.\ vergelijkende studie of risico-analyse)? Zorg er dan ook voor dat we duidelijk de verschillende stappen terug vinden die we verwachten in dit soort onderzoek!

Vermijd onderzoekstechnieken die geen objectieve, meetbare resultaten kunnen opleveren. Enquêtes, bijvoorbeeld, zijn voor een bachelorproef informatica meestal \textbf{niet geschikt}. De antwoorden zijn eerder meningen dan feiten en in de praktijk blijkt het ook bijzonder moeilijk om voldoende respondenten te vinden. Studenten die een enquête willen voeren, hebben meestal ook geen goede definitie van de populatie, waardoor ook niet kan aangetoond worden dat eventuele resultaten representatief zijn.

Uit dit onderdeel moet duidelijk naar voor komen dat je bachelorproef ook technisch voldoen\-de diepgang zal bevatten. Het zou niet kloppen als een bachelorproef informatica ook door bv.\ een student marketing zou kunnen uitgevoerd worden.

Je beschrijft ook al welke tools (hardware, software, diensten, \ldots) je denkt hiervoor te gebruiken of te ontwikkelen.

Probeer ook een tijdschatting te maken. Hoe lang zal je met elke fase van je onderzoek bezig zijn en wat zijn de concrete \emph{deliverables} in elke fase?

\vspace{1em}

%---------- Verwachte resultaten ----------------------------------------------
\section{Verwacht resultaat, conclusie}%
\label{sec:verwachte_resultaten}

Hier beschrijf je welke resultaten je verwacht. Als je metingen en simulaties uitvoert, kan je hier al mock-ups maken van de grafieken samen met de verwachte conclusies. Benoem zeker al je assen en de onderdelen van de grafiek die je gaat gebruiken. Dit zorgt ervoor dat je concreet weet welk soort data je moet verzamelen en hoe je die moet meten.

Wat heeft de doelgroep van je onderzoek aan het resultaat? Op welke manier zorgt jouw bachelorproef voor een meerwaarde?

Hier beschrijf je wat je verwacht uit je onderzoek, met de motivatie waarom. Het is \textbf{niet} erg indien uit je onderzoek andere resultaten en conclusies vloeien dan dat je hier beschrijft: het is dan juist interessant om te onderzoeken waarom jouw hypothesen niet overeenkomen met de resultaten.

\vspace{1em}
