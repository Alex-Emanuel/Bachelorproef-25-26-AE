%---------- Inleiding ---------------------------------------------------------

\section{Inleiding}%
\label{sec:inleiding}

De voorbije jaren wordt er steeds meer gestreamd in België. Populaire streamingdiensten zoals \emph{Netflix}, \emph{Disney+} en \emph{Spotify} zijn haast niet meer weg te denken. Daarnaast springt ook de Vlaamse entertainmentindustrie volop op deze trend. Enkele populaire Vlaamse zenders en productiehuizen kwamen de voorbije jaren met hun eigen platformen. Denk maar aan \emph{Streamz}, \emph{VRT MAX} en \emph{VTM GO}. Door het stijgende aanbod aan streamingdiensten en doordat live televisie in populariteit daalt, sluiten steeds meer Belgen, waaronder Vlamingen, streamingabonnementen af.

\vspace{1em}

Echter wat een groot aantal Belgen niet weet, is dat veel van deze platformen gebruikmaken van dark patterns. Het gaat hierbij om misleidende technieken die online gebruikers subtiel beïnvloeden om bepaalde keuzes te maken die ze anders niet zouden maken \autocite{FODEconomie2024}. Voorbeelden hiervan zijn onder andere moeilijk vindbare annulatieknoppen of vooraf aangevinkte selectievakjes. Niet enkel is het gebruik van deze patronen ethisch onverantwoord, het maakt gebruikers ook enorm kwetsbaar doordat ze deze patronen onvoldoende of niet opmerken. Daarnaast wordt de ernst van de gevolgen vaak onderschat, zoals het onbewust afsluiten van een abonnement of het betalen van een verborgen extra kost.

\vspace{1em}

Ondanks dat er veel onderzoek werd gedaan naar dark patterns vanuit het standpunt van zij die er gebruik van maken, zijn er weinig studies die de effecten van deze patronen op de gebruikers (lees \emph{slachtoffers}) onder de loep nemen. Daarbovenop is er nood aan onderzoek naar hoe de gebruikers hier meer bewust van kunnen worden. Uit deze bevindingen werd onderstaande onderzoeksvraag opgesteld:

\vspace{1em}

\begin{quote}
    \textit{``Welke digitale tool kan ontwikkeld worden om jongvolwassen Vlamingen tussen 21 en 25 jaar te helpen dark patterns op streamingdiensten sneller te herkennen, zodat onbewuste beïnvloeding voorkomen wordt?''}
\end{quote}

\vspace{1em}

Om de doelgroep van dit onderzoek te bepalen is er in eerste instantie gekeken naar de leeftijdsgroep die het meest actief gebruikmaakt van streamingdiensten in België. Uit de statistieken van \textcite{StatistiekVlaanderen2025} blijkt dat dit vooral jongvolwassenen zijn tussen 18 en 34 jaar. Daarnaast is de doelgroep verder vernauwd, om het onderzoek haalbaarder te maken. Hierdoor ligt de uiteindelijke focus op Vlaamse gebruikers tussen 21 en 25 jaar die actief gebruikmaken van streamingdiensten en interesse tonen in het afsluiten van een abonnement.

\vspace{1em}

Aanvullend werden enkele deelvragen geformuleerd die ondersteuning bieden bij het beter begrijpen van het probleem en het zoeken naar een mogelijke oplossing.

\vspace{1em}

\noindent
\textbf{Deelvragen probleemdomein:}
\begin{enumerate}
    \item Welke dark patterns komen het meest voor op websites en apps van streamingdiensten die door jongvolwassenen tussen 21 en 25 gebruikt worden?
    \item Hoe beïnvloeden deze dark patterns het beslissingsproces en de gebruikerservaring (vertrouwen, comfort, UX) van jongvolwassenen bij het afsluiten en opzeggen van streaming-abonnementen?
    \item In welke mate zijn jongvolwassenen zich bewust van de aanwezigheid van dark patterns bij het gebruik van streamingdiensten?
    \item Welke verbanden bestaan er tussen de aanwezigheid van specifieke dark patterns en de businessdoelstellingen van streaming-\newline diensten?
    \item Op welk platform (website of mobiele app) komen dark patterns het vaakst voor en waar kan een digitale tool de grootste impact hebben op gebruikersbewustzijn?
\end{enumerate}

\vspace{1em}

\noindent
\textbf{Deelvragen oplossingsdomein:}
\begin{enumerate}
    \item Welke wet- en regelgeving, richtlijnen of aanbevelingen bestaan er rond het gebruik van dark patterns in België en de Europese Unie?
    \item Welke bestaande tools die gebruikers waarschuwen voor dark patterns bestaan er al, en hoe effectief zijn deze in het verhogen van bewustzijn?
    \item Op welke manier kunnen jongvolwassenen het best ondersteund worden bij het herkennen van dark patterns zonder dat hun gebruikerservaring negatief beïnvloed wordt?
    \item Welke digitale of interactieve oplossing kan het meest bijdragen aan het verhogen van bewustzijn, en welke (niet-)functionele vereisten zijn hiervoor essentieel?
    \item In welke mate verhoogt het gebruik van de ontwikkelde digitale tool het bewustzijn van gebruikers over dark patterns, en welke verbeterpunten kunnen worden geïdentificeerd op basis van gebruikersfeedback?
\end{enumerate}

\vspace{1em}

Als uiteindelijk doel van dit onderzoek wordt er achterhaald welke digitale tool het meest geschikt is om gebruikers meer attent te maken op de aanwezigheid van dark patterns zodat ze hier bewuster van worden. Naast de scriptie wordt als concreet eindresultaat eveneens een Proof-of-Concept (PoC) uitgewerkt in de vorm van een tool die Vlamingen tussen 21 en 25 jaar tijdens het gebruik van streamingplatformen bewust maakt van en waarschuwt voor de aanwezigheid van dark patterns. Tot slot moet deze tool hen assisteren bij het maken van bewustere, weloverwogen keuzes bij onder meer het afsluiten en opzeggen van abonnementen.

\vspace{1em}

%---------- Stand van zaken ---------------------------------------------------

\section{Literatuurstudie}%
\label{sec:literatuurstudie}

\subsection{Definitie van dark patterns}
\label{subsec:definitie}

\vspace{1em}

Voor online verkoop een prominente rol speelde op de handelsmarkt, werden volgens \textcite{Kollmer2022} technieken voor manipulatie en misleiding voornamelijk toegepast in fysieke winkels. Echter werden deze methodes door de digitalisering ook overgebracht naar de wereld van e-commerce. Hierdoor ontstonden de zogenaamde dark patterns. Deze term deed dankzij Harry Brignull in 2010 zijn intrede, die deze patronen omschreef als \textbf{trucs} gebruikt op websites en apps \textbf{om gebruikers tegen hun wil dingen te laten doen}, zoals iets aankopen of zich ergens voor inschrijven \autocite{Kollmer2022}. 

\vspace{1em}

Bovendien kunnen deze patronen, zoals beschreven door \textcite{DiGeronimo2020}, verder onderverdeeld worden in vijf categorieën. Ten eerste is er \emph{nagging}, het herhaaldelijk vragen aan de gebruiker om iets anders te doen. Ten tweede \emph{obstruction}, het opzettelijk moeilijker maken van het uitvoeren van acties. Vervolgens bestaat \emph{sneaking}, het verbergen van relevante informatie zoals bijvoorbeeld verborgen kosten. Daarnaast is er sprake van \emph{interface interference}: het verwarrend, misleidend of slecht leesbaar maken van user interface elementen. Tot slot is er \emph{forced action} waarbij de gebruiker gedwongen wordt om ongewenste keuzes te maken.

\vspace{1em}

\subsection{Gebruik van dark patterns op digitale platformen}
\label{subsec:verspreiding}

\vspace{1em}

Uit een rapport van \textcite{ICPEN2024} is gebleken dat dark patterns voorkomen in een wijde selectie aan sectoren. Denk onder meer aan de entertainment-, streaming-, fitness- en e-commerce industrie. Eveneens bleek uit dit onderzoek dat zo’n 75,7\% van apps en websites die abonnementen verkopen minstens één dark pattern gebruiken. Aanvullend werd er geconstateerd dat bij 66,8\% van deze onderzochte websites en apps twee of meer patronen terug te vinden zijn. Verder concludeerden \textcite{Nygren2024} dat binnen de streamingsector, zowel de gratis als de te betalen platformen, uitgebreid gebruikmaken van deze patronen. Ook werd dankzij de studie van \textcite{Gunawan2021} duidelijk dat zowel de desktop- als mobiele versies dark patterns bevatten, maar dat de versies van éénzelfde dienst daarvoor niet meteen dezelfde patronen inzetten.


\vspace{1em}

\subsection{Bewustzijn van dark patterns en de effecten op gebruikersgedrag}
\label{subsec:effecten}

\vspace{1em}

Zoals de definitie van een dark pattern aanhaalt is zo’n patroon bedoelt om de gebruiker te manipuleren \autocite{Kollmer2022}. Maar wat is de invloed hiervan op het bewustzijn en gebruikersgedrag? \textcite{OECD2022} onderzocht dit en constateerde dat de meeste gebruikers moeite hebben bij het herkennen van deze technieken, terwijl sommige gebruikers ze wel opmerken maar er toch slachtoffer van worden. Volgens \textcite{BongardBlanchy2021} is een aanzienlijk deel van de gebruikers op de hoogte van het gebruik van deze patronen en verwachten ze het bij bekende diensten zoals bijvoorbeeld Amazon en Netflix. Echter beseffen ze niet de mate waarin deze patronen schade kunnen aanrichten. Ook zij halen aan dat het herkennen lastig blijft, waardoor gebruikers hun nog altijd laten beïnvloeden, zelfs als ze weten dat deze praktijken worden toegepast. Hieruit konden ze concluderen dat gebruikers zich niet bewust zijn van hoe deze ontwerpkeuzes hun concreet kunnen schaden \autocite{BongardBlanchy2021}. 

\vspace{1em}

Aanvullend werd door \textcite{OECD2022} onderzocht welke patronen het gedrag het meest beïnvloeden. Hieruit konden ze 2 types onderscheiden: de “\emph{mild}” en “\emph{aggressive}” dark patterns. Volgens hen zijn de “\emph{mild}” dark patterns (de minder opvallende patronen) vaak effectiever doordat ze zo moeilijk op te merken zijn. Verder werd door \textcite{OECD2022} onderzocht wat de effecten zijn op de gebruiker. Zo schaadt het de autonomie, kan het leiden tot financieel verlies, als ook zorgen de technieken voor privacy schending en kunnen gebruikers emotionele stress en tijdverlies ervaren. 


\vspace{1em}

\subsection{De ethiek achter het gebruik van dark patterns}
\label{subsec:ethisch-ontwerp}

\vspace{1em}

Hoewel niet alle dark patterns manipulatief zijn \autocite{Kitkowska2023}, blijven er enkele ethische aspecten waar over nagedacht moet worden. Dark patterns spelen namelijk in op de gewoontes en het denken van de gebruikers. Zo maken ze gebruik van de angst om iets te missen en van de voorkeur van gebruikers voor de eenvoudigste keuze. \textcite{DiGeronimo2020} stelt dat een bruikbare applicatie niet per se duidt op een ethische applicatie. Volgens hen hebben ontwerpers de rol om met hun ontwerp de gebruiker te overtuigen terwijl die macht vaak misbruikt wordt. Tot dusver is er volgens deze studie nog geen éénduidige definitie voor wat een ethische user interface juist inhoudt, echter hebben de meeste experts een gelijkaardige mening: een user interface wordt als ethisch beschouwd wanneer het ontwerp meer bijdraagt tot het leven van de gebruiker dan dat het de gebruiker schaadt. Dit brengt echter volgende vraag met zich mee: \emph{``Op welk punt gaat een ontwerp van het ondersteunen over naar het misleiden van de gebruiker?''}. Tot slot is het volgens \textcite{Vandenberghe2016} belangrijk om user-centered design niet meteen gelijk te stellen aan ethisch ontwerp. Bepaalde methoden kunnen namelijk nog steeds schadelijke effecten hebben. Volgens hen bestaat er een \emph{Ethical Design Manifesto} dat benadrukt dat een ontwerp niet enkel moet verleiden of esthetisch zijn, maar het gebruikers ook kritisch moet laten nadenken over de impact van technologie op hun leven.

\vspace{1em}

\subsection{Onderzoek en tools met betrekking tot dark patterns}
\label{subsec:huidig-onderzoek}

\vspace{1em}

In het algemeen werd er in het kader van dark patterns voornamelijk onderzocht welke verschillende soorten dark patterns bestaan en waar deze het frequentst voorkomen \autocite{Luguri2021}. Naast deze studies is er volgens \textcite{Luguri2021} nog te weinig onderzoek gedaan op andere gebieden binnen dit domein. Ook de studie van \textcite{DiGeronimo2020} benadrukt de nood naar verder onderzoek. Volgens hen is het belangrijk dat de focus wordt gelegd op het vinden van manieren om bewustwording bij gebruikers te stimuleren, onder meer via de ontwikkeling van geautomatiseerde en educatieve tools. Hoewel er volgens \textcite{Nie2024} al redelijk wat detectietools bestaan die door middel van AI dark patterns opsporen, bieden deze een te lage dekking.  \textcite{Nie2024} identificeerden maar liefst 64 types dark patterns, waarvan slechts de helft werd gedetecteerd door de acht onderzochte geautomatiseerde tools. 

\vspace{1em}

%---------- Methodologie ------------------------------------------------------
\section{Methodologie}%
\label{sec:methodologie}

Hier beschrijf je hoe je van plan bent het onderzoek te voeren. Welke onderzoekstechniek ga je toepassen om elk van je onderzoeksvragen te beantwoorden? Gebruik je hiervoor literatuurstudie, interviews met belanghebbenden (bv.~voor requirements-analyse), experimenten, simulaties, vergelijkende studie, risico-analyse, PoC, \ldots?

Valt je onderwerp onder één van de typische soorten bachelorproeven die besproken zijn in de lessen Research Methods (bv.\ vergelijkende studie of risico-analyse)? Zorg er dan ook voor dat we duidelijk de verschillende stappen terug vinden die we verwachten in dit soort onderzoek!

Vermijd onderzoekstechnieken die geen objectieve, meetbare resultaten kunnen opleveren. Enquêtes, bijvoorbeeld, zijn voor een bachelorproef informatica meestal \textbf{niet geschikt}. De antwoorden zijn eerder meningen dan feiten en in de praktijk blijkt het ook bijzonder moeilijk om voldoende respondenten te vinden. Studenten die een enquête willen voeren, hebben meestal ook geen goede definitie van de populatie, waardoor ook niet kan aangetoond worden dat eventuele resultaten representatief zijn.

Uit dit onderdeel moet duidelijk naar voor komen dat je bachelorproef ook technisch voldoen\-de diepgang zal bevatten. Het zou niet kloppen als een bachelorproef informatica ook door bv.\ een student marketing zou kunnen uitgevoerd worden.

Je beschrijft ook al welke tools (hardware, software, diensten, \ldots) je denkt hiervoor te gebruiken of te ontwikkelen.

Probeer ook een tijdschatting te maken. Hoe lang zal je met elke fase van je onderzoek bezig zijn en wat zijn de concrete \emph{deliverables} in elke fase?

\vspace{1em}

%---------- Verwachte resultaten ----------------------------------------------
\section{Verwacht resultaat, conclusie}%
\label{sec:verwachte_resultaten}

Hier beschrijf je welke resultaten je verwacht. Als je metingen en simulaties uitvoert, kan je hier al mock-ups maken van de grafieken samen met de verwachte conclusies. Benoem zeker al je assen en de onderdelen van de grafiek die je gaat gebruiken. Dit zorgt ervoor dat je concreet weet welk soort data je moet verzamelen en hoe je die moet meten.

Wat heeft de doelgroep van je onderzoek aan het resultaat? Op welke manier zorgt jouw bachelorproef voor een meerwaarde?

Hier beschrijf je wat je verwacht uit je onderzoek, met de motivatie waarom. Het is \textbf{niet} erg indien uit je onderzoek andere resultaten en conclusies vloeien dan dat je hier beschrijft: het is dan juist interessant om te onderzoeken waarom jouw hypothesen niet overeenkomen met de resultaten.

\vspace{1em}
